\section{\label{conclusions}Conclusions}


We examined in detail the evacuation of pedestrians for the situation where 
two contiguous doors are available for leaving the room. Throughout  
Section~\ref{results} we presented results on the evacuation performance under 
high anxiety levels and increasing number of pedestrians. Both conditions 
exhibit the novel result that a worsening in the evacuation time exists as the door 
separation distance $d_g$ increases from the null value to roughly the width of 
two pedestrians. Special situations may enhance the evacuation performance for 
larger values of $d_g$. \\

The range from $d_g=0$ to $d_g\gg d_w$ was inspected. 
In the interval $0\leq d_g \leq 2r_{ij}$ (two pedestrian's width), the evacuation 
performance worsened for all the explored situations, as the separation
distance between doors $d_g$ increased. But, from $d_g > 2r_{ij}$ the 
evacuation time enhanced for relatively small crowds and moderate anxiety
levels. We realized that the sharp change in the evacuation behaviour at
$d_g=2r_{ij}$ corresponded to qualitative differences in 
the pedestrian dynamics close to the exits.\\

After a detailed comparison of the dynamics for the single door situation and 
for two doors very close to each other (that is, $d_g<2r_{ij}$), we concluded that 
the blocking structures (\emph{i.e.} blocking arcs) around the openings were 
released intermittently, allowing the pedestrians to leave the room in a 
stop-and-go process. As the separation distance approached $2r_{ij}$, the 
blocking arcs around each door, resembled the blocking 
situation of two single doors. This changes only affected the 
local dynamics (close to the doors), while the crowd remained gathered into a 
single clogging area. \\

For $d_g>2r_{ij}$ the single door blocking structures become relevant even for 
large values of $d_g$ (see Fig.~\ref{fig:14}). No further qualitative changes 
were observed locally around each door. However, increasing the crowd size ($N$) 
or the pedestrian's anxiety level ($v_d$) slowed down the evacuation. Both 
magnitudes are linked to the pressure acting on the pedestrians, and therefore, 
enhanced the ``faster is slower'' effects. \\

For a better understanding of the relationship between $N$, $v_d$ and the 
pressure in the bulk, a simple lane example complemented our analysis. It was 
shown that the classical virial expression is still suitable for the 
investigation of social systems.  \\


