Current regulations demand that at least two exits should be available 
for a safe evacuation during a panic situation. The second exit is  
expected to reduce the overall clogging, and  consequently, improve the 
evacuation time. However, rooms having contiguous doors not always reduce the 
leaving time as expected. We investigated the relation between the door's 
separation and the evacuation performance. We found that there exists a 
separation distance range that does not really improve the evacuation time, or 
it can even worsen the process performance. To our knowledge, no attention has 
been given to this issue in the literature. This work reports how the 
pedestrian's dynamics differ when the separation distance between two exit doors 
changes and how this affects the overall performance.