\section{\label{background}Background}

\subsection{The Social Force Model}

The ``social force model'' (SFM) deals with the pedestrians behavioural 
pattern in a crowded environment. The basic model states that the 
pedestrians motion is controlled by three kind of forces: the ``desire force'', 
the ``social force'' and the ``granular force''. The three are very different in 
 nature, but enter into an equation of motion as follows  \\ 


\begin{equation}
m_i\,\displaystyle\frac{d\mathbf{v}^{(i)}}{dt}(t)=\mathbf{f}_d^{(i)}
(t)+\displaystyle\sum_{j}\displaystyle\mathbf{f}_s^{(ij)}(t)+\displaystyle\sum_{
j}\mathbf{f}_g^{(ij)}(t)\label{eqn_1}
\end{equation}

\noindent where $m_i$ is the mass of the pedestrian $i$, and $\mathbf{v}_i$ is 
its corresponding velocity. The subscript $j$ represents all other pedestrians 
(excluding $i$) and the walls. $\mathbf{f}_d$, $\mathbf{f}_s$ and 
$\mathbf{f}_g$ are the desire force, the social force and the granular force, 
respectively. See Refs.~\cite{Helbing1,Dorso1,Dorso2,Dorso3,Dorso4} for 
details.\\

The desire force reflects the pedestrian's own desire to go to a specific 
place \cite{Helbing1}. He (she) needs to accelerate (decelerate) from his (her) 
current velocity, in order to achieve his (her) own willings. As he (she) 
reaches the velocity that makes him (her) feel comfortable, no further 
acceleration (deceleration) is required. This velocity is the ``desired 
velocity'' of the pedestrian $\mathbf{v}_d(t)$. The expression for 
$\mathbf{f}_d$ in Eq.~(\ref{eqn_2}) handles this issue.  \\


\begin{equation}
\left\{\begin{array}{lcl}
        \mathbf{f}_d^ {(i)}(t) & = & m_i\,\displaystyle\frac{\mathbf{v}_d^
{(i)}(t)-\mathbf{v}_i(t)}{\tau} \\
        & & \\
\mathbf{f}_s^{(ij)} & = & A_i\,e^{(r_{ij}-d_{ij})/B_i}\mathbf{n}_{ij}\\
        & & \\
\mathbf{f}_g^{(ij)} &= &\kappa\,g(r_{ij}-d_{ij})\,\Delta
\mathbf{v}_{ij}\cdot\mathbf{t}_{ij} \\
       \end{array}\right.\label{eqn_2}
\end{equation}

$\tau$ means a relaxation time. Further details on each parameter can be found 
in Refs.~\cite{Helbing1,Dorso1,Dorso2,Dorso3,Dorso4}.\\

Notice that the desired velocity $\mathbf{v}_d$ has magnitude $v_d$ and points 
to the desired place at the direction $\hat{\mathbf{e}}_d$. Thus, $v_d$ 
represents his (her) state of anxiety, white $\hat{\mathbf{e}}_d$ indicates the 
place where he (she) is willing to go. We assume, for simplicity, that 
$v_d$ remains constant during an evacuation process, but $\hat{\mathbf{e}}_d$ 
changes according to the current position of the pedestrian.   \\

The social force $\mathbf{f}_s$ corresponds to the tendency of each individual 
to keep some space between him and other pedestrians, or, between him and the 
walls \cite{Helbing4}. The $\mathbf{f}_s$ expressed in Eq.~(\ref{eqn_2}) 
depends on the inter-pedestrian distance $d_{ij}$. The magnitude 
$r_{ij}=r_i+r_j$ is the sum of the pedestrian's radius, while $A_i$ and $B_i$ 
are two fixed parameters ($r_j=0$ for the interaction with the wall). Thus, 
$\mathbf{f}_s$ is a repulsive monotonic force that resembles the pedestrian 
feelings for preserving his (her) \textit{private sphere} 
\cite{Helbing1,Helbing4}. \\

The granular force $\mathbf{f}_g$ appearing in Eq.~(\ref{eqn_1}) represents the 
sliding friction between contacting people (or between people  and walls). Its 
expression can be seen also in Eq.~(\ref{eqn_2}). It is assumed to be a linear 
function of the relative (tangential) velocities $\Delta
\mathbf{v}_{ij}\cdot\mathbf{t}_{ij}$ of the contacting individuals. The 
function $g(r_{ij}-d_{ij})$ returns the argument value if $r_{ij}>d_{ij}$, 
while $\kappa$ is a fixed parameter (see 
Refs.~\cite{Helbing1,Dorso1,Dorso2,Dorso3,Dorso4}).\\

{\color{red} One of the most remarkable phenomena attained by this model is the ``faster is slower'' effect. It states that the higher the desired velocity, the higher the evacuation time~\cite{Helbing1}. Experimental data has achieved this effect, while the pressure of the bulk raises as a relevant magnitude in the worsening of the evacuation time. Experiments also show that the bulk pressure is a function of the number  of people and the maximum group speed. The latter is associated to the desired velocity in the Social Force Model (see Ref.~\cite{Pastor}).}  
 


\subsection{\label{human}Clustering structures}

The time delays during an evacuation process are related to clustering people 
as explained in Refs.~\cite{Dorso1,Dorso2}. Groups of pedestrians can 
be defined as the set of individuals that for any member of the group (say, 
$i$) there exists at least another member belonging to the same group ($j$) 
in contact with the former. That is, 

\begin{equation}
 i\in\mathcal{G} \Leftrightarrow \exists j\in\mathcal{G}/d_{ij}<r_i+r_j
\end{equation}

\noindent where $\mathcal{G}$ corresponds to any set of individuals. 
 This kind of structure is called a \textit{human cluster}. \\

From all human clusters appearing during the evacuation process, those that 
are simultaneously in contact with the walls on both sides of the exit are 
the ones that possibly \textit{block} the way out. Thus, we are interested 
in the minimum number of contacting pedestrians belonging to this 
\textit{blocking cluster} that are able to link both sides of the exit. We call 
this minimalistic group as a \textit{blocking structure}. Any blocking structure 
is supposed to work as a barrier for the pedestrians in behind.   \\



\subsection{\label{pressure}The local pressure on the pedestrians}

The pressure on a single pedestrian (say, $i$) is defined as \cite{Helbing1}

\begin{equation} 
P_i=\displaystyle\frac{1}{2\pi 
r_i}\displaystyle\sum_{j=1}^{N-1}\mathbf{f}_s^{(ij)}\cdot\mathbf{n}_{ij}
\label{eqn_4b}
\end{equation}

$\mathbf{f}_{s}^{(ij)}$ are the forces acting on the individual $i$ due to the 
other individuals. Recall that these forces point from any individual $j$ to 
the individual $i$, and thus, the products 
$\mathbf{f}_{s}^{(ij)}\cdot\mathbf{n}_{ij}$ are always positive. \\ 

Notice that Eq.~(\ref{eqn_4b}) holds either if the pedestrians are in 
contact or not. The feelings for preserving the \textit{private sphere} 
actuate as a ``social pressure'' that makes possible for the individuals to 
change their behavioural pattern when they come too close to each other or to 
the walls.\\

A more formal definition for the ``social pressure'' is given in
\ref{alternative}. We show that the definition (\ref{eqn_4b}) is in accordance with the 
one in \ref{alternative}, if the momentum $p_i$ of the individuals become 
neglectable. Thus, the expression (\ref{eqn_4b}) is suitable for clogging 
situations where the pedestrians move slowly. \\

We further applied the formal definition for the ``social pressure'' to a 
simple example in \ref{app}. We also checked that both definitions 
give the same results all through Section \ref{results}. \\

