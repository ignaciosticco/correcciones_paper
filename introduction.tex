\section{\label{introduction}Introduction}

The practice of providing two doors for emergency evacuation can be traced back 
to the last Qing dynasty in China (1644-1911 AD). A mandatory regulation 
established that large buildings had to provide two fire exits \cite{cheng}. 
This kind of regulations upgraded to current standard codes with detailed  
specifications on the exits position, widths and separations \cite{OSHA,FLO}. \\


Current regulations claim that the minimum door width should be 0.813~m while 
the maximum door-leaf should not exceed 1.219~m \cite{FLO,FLO2}. If more than 
two doors are required, the distance between two of them must be at least 
one-half or one-third of the room diagonal distance. But, no special 
requirements apply to the rest of the doors.  \\

The rulings leave some space for placing the extra openings (\emph{i.e.} those 
above two exits) at an arbitrary separation distance. Thus, it is possible to 
place a couple of doors on the same side of the room at any distance. The 
special case of two contiguous doors has been examined throughout the 
literature \cite{kirchner1,perez1,daoliang1,huanhuan1}.  \\

Kirchner and Schadschneider studied the pedestrians evacuation process through 
two contiguous doors using a cellular automaton model \cite{kirchner1}. The 
agents were able to leave the room under increasing panic situations for 
behavioural patterns varying from individualistic pedestrians to strongly 
coupled pedestrians moving like a \emph{herd}. The evacuation time was found 
to be independent of the separation distance between doors for the 
individualistic pedestrians in a panic situation. But if the pedestrians were 
allowed to move like a herd, an increasing evacuation time for small separation 
lengths (less than 10 individuals size) was reported. \\

The above conclusions are not in complete agreement with the investigation 
acknowledged in Ref.~\cite{perez1}. The authors assert that the total number of 
pedestrians leaving the room per unit time slows-down for separation 
distances (between doors) smaller than four door widths \cite{perez1}. This 
slow-down is identified as a disruptive interference effect due to pedestrians 
crossing in each other's path.  For the particular case analyzed in this work,  
the threshold of four door widths ($4\,d_w$) corresponds to the distance 
separation necessary to distinguish two independent groups of pedestrians, each 
one surrounding the nearest door. \\

Researchers called the attention on the fact that no matter how separated 
the two contiguous doors are placed, the overall performance does not improve 
twice with respect to a single exit (of the same total width). This 
effect is attributed to some sort of pedestrian interference  \cite{perez1}. 
\\

Although the above results were obtained for very narrow doors (\emph{i.e.} 
single individual width), further investigation showed that they also apply 
to doors allowing two simultaneous leaving pedestrians. However, this does 
not hold for a room with a single door \cite{daoliang1}. In this case, it is 
true that the mean flux of evacuating people increases with an increasing door 
width, but the ratio flux per door width decreases \cite{Dorso1}.   \\

It was observed in Ref.~\cite{kirchner1,daoliang1} that the two contiguous 
doors should not be placed near the wall corners, since the side walls affect 
negatively the evacuation efficiency. No further explanation was given on this 
phenomenon, although the authors concluded this may cause a worsening in the 
evacuation performance for large separation distances between doors.  \\


A recent investigation (Ref.~\cite{huanhuan1}) on evacuation processes of 
cellular automata suggests that five distances should be taken into account 
when studying the evacuation performance: the total width of the openings (that 
is, adding the widths of each door), the doors separation distance, the width 
difference between the two doors, and the distance to the nearest corner.   \\

From the results shown in Ref.~\cite{huanhuan1}, the evacuation time depends on 
the total width of the openings (if both doors have the same width). But, for 
a fixed total width of the opening, it appears that the optimal location of the 
exits depends on the doors separation distance. \\

Our investigation focuses on symmetric configurations with equally sized 
doors. At variance to the above mentioned literature, we examine the evacuation 
dynamics by means of the Social Force Model (SFM). An overview of this model 
can be found in Section \ref{background}. \\

In Section \ref{simulations} we describe the specific settings for the 
evacuation processes. The measurement conditions for the simulations can also 
be found there. \\

In Sections \ref{faster_is_slower} to \ref{null_gap_patterns} the single door 
configuration is revisited. Its purpose is to make easier the understanding of 
the two-doors configuration for very small separation distances $d_g$. \\

In Section \ref{door_seperation} we examine the case of two separated doors. 
We explore the effect of increasing the separation distance $d_g$ until the 
clogging areas close to each door become almost independent. \\

Section \ref{conclusions} resumes the pedestrians behavioural patterns, and its 
consequences on the evacuation performance, for the different door 
separation scenarios. \\  


