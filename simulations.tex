\section{\label{simulations}Numerical simulations}



\subsection{\label{numerical_geometry} Geometry and process simulation}


We simulated different evacuation processes for room sizes of 20~m $\times$ 
20~m, 30~m $\times$ 30~m and 40~m $\times$ 40~m. The rooms had one or two exit 
doors on the same wall, as shown in Fig.~\ref{fig:19}. The doors were placed 
symmetrically from the mid position of the wall, in order to avoid corner 
effects. Both doors had also the same width.  \\

\begin{figure}
\includegraphics[width=\columnwidth]{./fig0.eps}
\caption{\label{fig:19} Snapshot of an evacuation process from a 
$20\,\mathrm{m}\times20\,$m room, with two doors. In red we can see a blocking 
structure around the upper door. The desired velocity was $v_d=4\,$m/s.  }
%  done by sticco
\end{figure}


At the beginning of the process, the pedestrians were all equally separated 
in a square arrangement. The occupancy density was initially set to 
0.6~people/m$^2$, close to the allowed limiting values by current regulations 
\cite{mysen}. They all had random velocities resembling a Gaussian distribution 
with null mean value. The pedestrians were willing to go to the nearest exit. 
Thus, all the pedestrians had the desired velocity $\mathbf{v}_d$ pointing to 
the same exit door if only one door was available, or to the nearest door if two 
exits were available.  \\

In order to focus on the effects due to dual exits, we only allowed the 
pedestrians to move individualistically, that is, neither leaderships nor 
herding behaviors were present during the evacuation process. At any time, the 
pedestrians knew the doors location and tried to escape by their own.  \\

The simulations were supported by {\sc Lammps} molecular 
dynamics simulator with parallel computing capabilities \cite{plimpton}. The 
time integration algorithm followed the velocity Verlet scheme with a time step 
of $10^{-4}\,$s. All the necessary parameters were set to the same values as in 
previous works (see Refs.~\cite{Dorso3,Dorso4}). It was assumed that all the 
individuals had the same radius ($r_i=0.3$~m) and weight ($m_i=70$~kg). We ran 
30 processes for each panic situation, in order to get enough data for mean 
values computation. \\

Although the {\sc Lammps} simulator has the most common built-in functions, 
neither the social force $\mathbf{f}_s$  nor the desire force $\mathbf{f}_d$ 
were available. We implemented special modules (with parallel computing 
compatibilities) for the $\mathbf{f}_s$ and $\mathbf{f}_d$ computations. 
These computations were checked over with previous computations. \\

The pedestrian's desired direction $\hat{\mathbf{e}}_d$ was updated at each 
time step. After leaving the room, they continued moving away. No re-entering 
mechanism was allowed. \\


\subsection{\label{numerical_data}Measurements conditions}

Simulations were run in the same way as in Refs. \cite{Dorso3,Dorso4}.  Each 
process started with all the individuals inside the room. The measurement period 
lasted until 80\% of the occupants left the room. If this condition could not be 
fulfilled within the first 3000~s, the process was stopped. Data was 
recorded at time intervals of $0.05\,$s (cf. Eq.~(\ref{eqn_2}a)).\\

The simulations ran from relaxed situations ($v_d<2\,$m/s) to very stressing 
rushes ($v_d=8\,$m/s). We registered the individuals positions and 
velocities for each evacuation process. Thus, we were able to compute the 
``social pressure'' through out the process and to trace the pedestrians 
behavioural pattern.  \\

%{\red The ``mean'' social pressures through out the process were computed as follows: the room was binned into squares of 1~m$\times$1~m. The cumulative social pressure for each bin was recorded every $0.05\,$s for all the processes. The cumulative number of pedestrians in each bin was also recorded. Mean values were computed as the ratio of both magnitudes.\\}

%{\red The mean flow of pedestrians was computed on a 1~m$\times$1~m bined space and following the same procedure as stated above for the mean social pressure.} 



